\documentclass[
    12pt,
    twoside,
    titlepage
]{article}

% https://tug.org/FontCatalogue/
% koristi sans serif font:
\usepackage[defaultsans]{droidsans}
\renewcommand*\familydefault{\sfdefault} %% Only if the base font of the document is to be typewriter style
\usepackage[T1]{fontenc}

\usepackage[
    a4paper,
    nomarginpar,
    margin=2.54cm,
    inner=3.54cm
]{geometry}

\usepackage[croatian]{babel}

\usepackage{blindtext} % debug: generiraj zamjenski tekst (Lorem ipsum...)

\newcommand{\accentColor}{DarkBlue} % akcent boja
% https://www.w3.org/TR/SVG11/types.html#ColorKeywords
% korištenje imenovanih boja (SVG 1.1 standard, osjetljiv na velika/mala slova):
\usepackage[svgnames]{xcolor}

\usepackage{fancyhdr}
\pagestyle{fancy}
\fancyhf{} % reset default header and footer
\renewcommand{\headrulewidth}{0pt}
\fancyfoot[LE,RO]{\thepage}

\usepackage[useregional]{datetime2}
\date{\DTMdisplaydate{2023}{10}{02}{-1}} % (-1 = ne prikazuj dan u tjednu)

\usepackage[dotinlabels]{titletoc} % točka poslije brojeva naslova u sadržaju
\usepackage{titlesec}
\AddToHook{cmd/section/before}{\clearpage}
\titleformat{\section}
    {\large\bfseries\color{\accentColor}}{\thesection.}{5pt}{}[{\titlerule[1pt]}]
\titleformat{\subsection}
    {\bfseries\color{\accentColor}}{\thesubsection.}{5pt}{}
\titleformat{\subsubsection}
    {\bfseries}{\thesubsubsection.}{5pt}{}

% \usepackage{showframe} % debug: prikaži obrube

\linespread{1.25}
\usepackage{indentfirst}

\usepackage{hyperref} % URL's
\hypersetup{
    colorlinks=true,
    linkcolor=black,
    urlcolor=DarkBlue        % color of external links
}
\urlstyle{same}

% \setcounter{tocdepth}{2}

\usepackage{graphicx}
\graphicspath{ {./images/} }

% \usepackage[skip=-10pt]{caption} % example skip set to 2pt

\usepackage{subcaption}
\DeclareCaptionFormat{custom}
{%
    \textbf{#1#2}\textit{\small #3}
}
\captionsetup{format=custom}

\usepackage{listings}

\usepackage{noto}
\usepackage[T1]{fontenc}

\lstdefinestyle{mystyle}{
    commentstyle=\color{DodgerBlue},
    keywordstyle=\bfseries\color{DarkBlue},
    % numberstyle=\color{Blue},
    stringstyle=\color{MediumBlue},
    basicstyle=\linespread{0.8}\ttfamily\footnotesize,
    breakatwhitespace=false,
    breaklines=true,
    captionpos=b,
    keepspaces=true,
    % numbers=left,
    showspaces=false,
    showstringspaces=false,
    showtabs=false,
    tabsize=2,
    frame=single
}

\lstset{style=mystyle}
\lstset{extendedchars=true, inputencoding=utf8,
literate      =        % Support additional characters
{á}{{\'a}}1  {é}{{\'e}}1  {í}{{\'i}}1 {ó}{{\'o}}1  {ú}{{\'u}}1
{Á}{{\'A}}1  {É}{{\'E}}1  {Í}{{\'I}}1 {Ó}{{\'O}}1  {Ú}{{\'U}}1
{à}{{\`a}}1  {è}{{\`e}}1  {ì}{{\`i}}1 {ò}{{\`o}}1  {ù}{{\`u}}1
{À}{{\`A}}1  {È}{{\`E}}1  {Ì}{{\`I}}1 {Ò}{{\`O}}1  {Ù}{{\`U}}1
{ä}{{\"a}}1  {ë}{{\"e}}1  {ï}{{\"i}}1 {ö}{{\"o}}1  {ü}{{\"u}}1
{Ä}{{\"A}}1  {Ë}{{\"E}}1  {Ï}{{\"I}}1 {Ö}{{\"O}}1  {Ü}{{\"U}}1
{â}{{\^a}}1  {ê}{{\^e}}1  {î}{{\^i}}1 {ô}{{\^o}}1  {û}{{\^u}}1
{Â}{{\^A}}1  {Ê}{{\^E}}1  {Î}{{\^I}}1 {Ô}{{\^O}}1  {Û}{{\^U}}1
{œ}{{\oe}}1  {Œ}{{\OE}}1  {æ}{{\ae}}1 {Æ}{{\AE}}1  {ß}{{\ss}}1
{ẞ}{{\SS}}1  {ç}{{\c{c}}}1 {Ç}{{\c{C}}}1 {ø}{{\o}}1  {Ø}{{\O}}1
{å}{{\aa}}1  {Å}{{\AA}}1  {ã}{{\~a}}1  {õ}{{\~o}}1 {Ã}{{\~A}}1
{Õ}{{\~O}}1  {ñ}{{\~n}}1  {Ñ}{{\~N}}1  {¿}{{?`}}1  {¡}{{!`}}1
{°}{{\textdegree}}1 {º}{{\textordmasculine}}1 {ª}{{\textordfeminine}}1
{£}{{\pounds}}1  {©}{{\copyright}}1  {®}{{\textregistered}}1
{«}{{\guillemotleft}}1  {»}{{\guillemotright}}1  {Ð}{{\DH}}1  {ð}{{\dh}}1
{Ý}{{\'Y}}1    {ý}{{\'y}}1    {Þ}{{\TH}}1    {þ}{{\th}}1    {Ă}{{\u{A}}}1
{ă}{{\u{a}}}1  {Ą}{{\k{A}}}1  {ą}{{\k{a}}}1  {Ć}{{\'C}}1    {ć}{{\'c}}1
{Č}{{\v{C}}}1  {č}{{\v{c}}}1  {Ď}{{\v{D}}}1  {ď}{{\v{d}}}1  {Đ}{{\DJ}}1
{đ}{{\dj}}1    {Ė}{{\.{E}}}1  {ė}{{\.{e}}}1  {Ę}{{\k{E}}}1  {ę}{{\k{e}}}1
{Ě}{{\v{E}}}1  {ě}{{\v{e}}}1  {Ğ}{{\u{G}}}1  {ğ}{{\u{g}}}1  {Ĩ}{{\~I}}1
{ĩ}{{\~\i}}1   {Į}{{\k{I}}}1  {į}{{\k{i}}}1  {İ}{{\.{I}}}1  {ı}{{\i}}1
{Ĺ}{{\'L}}1    {ĺ}{{\'l}}1    {Ľ}{{\v{L}}}1  {ľ}{{\v{l}}}1  {Ł}{{\L{}}}1
{ł}{{\l{}}}1   {Ń}{{\'N}}1    {ń}{{\'n}}1    {Ň}{{\v{N}}}1  {ň}{{\v{n}}}1
{Ő}{{\H{O}}}1  {ő}{{\H{o}}}1  {Ŕ}{{\'{R}}}1  {ŕ}{{\'{r}}}1  {Ř}{{\v{R}}}1
{ř}{{\v{r}}}1  {Ś}{{\'S}}1    {ś}{{\'s}}1    {Ş}{{\c{S}}}1  {ş}{{\c{s}}}1
{Š}{{\v{S}}}1  {š}{{\v{s}}}1  {Ť}{{\v{T}}}1  {ť}{{\v{t}}}1  {Ũ}{{\~U}}1
{ũ}{{\~u}}1    {Ū}{{\={U}}}1  {ū}{{\={u}}}1  {Ů}{{\r{U}}}1  {ů}{{\r{u}}}1
{Ű}{{\H{U}}}1  {ű}{{\H{u}}}1  {Ų}{{\k{U}}}1  {ų}{{\k{u}}}1  {Ź}{{\'Z}}1
{ź}{{\'z}}1    {Ż}{{\.Z}}1    {ż}{{\.z}}1    {Ž}{{\v{Z}}}1 {ž}{{\v{z}}}1
{Ž}{{\v{Z}}}1
% ¿ and ¡ are not correctly displayed if inconsolata font is used
% together with the lstlisting environment. Consider typing code in
% external files and using \lstinputlisting to display them instead.
}
\lstset{inputpath="./code/"}
\renewcommand{\lstlistingname}{Izvorni kod}% Listing -> Algorithm
\renewcommand{\lstlistlistingname}{Popis izvornih kodova}% List of Listings -> List of Algorithms

\lstdefinelanguage{JavaScript}{
  keywords={break, case, catch, continue, debugger, default, delete, do, else, finally, for, function, if, in, instanceof, new, return, switch, this, throw, try, typeof, var, void, while, with},
  morecomment=[l]{//},
  morecomment=[s]{/*}{*/},
  morestring=[b]',
  morestring=[b]",
  sensitive=true
}

\usepackage[nottoc]{tocbibind} % To include lists in the TOC
\usepackage{wrapfig}

\usepackage{pdfpages}

\usepackage{tabularx}

\usepackage{calc}

\newlength{\mytextwidth}
\newlength{\myline}


\begin{document}

\begin{titlepage}
    \begin{flushleft}
        Srednja škola Krapina\\
        Šetalište hrvatskog narodnog preporoda 6\\
        49000 Krapina, Hrvatska\\
    \end{flushleft}

    \vfill

    \begin{center}
        \LARGE
        \textbf{
            Web aplikacija za izradu, rješavanje i statistiku ispita i obrazaca}\\

        \vspace{0.5cm}

        \Large
        \textcolor{\accentColor}{\textbf{Elaborat završnog rada}}\\

    \end{center}

    \vfill

    \begin{flushleft}
        \normalsize
        \textbf{Zanimanje:} tehničar za računalstvo\\
        \textbf{Predmet:} Napredno i objektno programiranje\\
        \textbf{Mentor:} Stjepan Šalković, mag. inf. univ. spec. oec.\\
        \textbf{Učenik:} Ivan Dolovčak, 4.AT\\
    \end{flushleft}

    \vspace{1cm}

    \begin{center}
        \today
    \end{center}
\end{titlepage}


\tableofcontents

\newpage

\section{Uvod}

  \subsection{Problematika}

    Ispiti i obrasci su svakodnevnica obrazovnog i poslovnog okruženja. Napretkom
    tehnologije i digitalizacijom pojavila se ideja za automatiziranjem procesa
    izrade profesionalnih obrazaca i ispita. Tako su nastale prve (web) aplikacije
    za izradu, ali i obradu takvih dokumenata. Takve aplikacije su bile
    inspiracija za temu ovog završnog rada.

  \subsection{Tema}

    Tema završnog rada je web aplikacija za izradu, rješavanje i statistiku
    ispita i obrazaca. Nadalje u tekstu riječ \textit{dokument} se odnosi na
    obrase i na ispite. Aplikacijom je moguće izraditi dokumente s određenim
    postavkama: tip dokumenta (obrazac ili ispit), rok predaje, broj dozvoljenih
    pokušaja predaje i vidljivost dokumenta (javni, privatni ili skriveni).
    Nakon izrade dokumenta moguće je i mijenjati te postavke. Također je moguće
    i obrisati dokumente.

    Aplikacija pruža različite prikaze popisa dokumenata. Moguće je vidjeti
    vlastite dokumente, ali i pretražiti tuđe, te ih i rješiti (predati).

    Da bi krajnji korisnik/ca aplikacije mogao/la pristupiti svim mogućnostima
    aplikacije, potrebno se registrirati i prijaviti u aplikaciju. Korisnik/ca
    nakon prijave može mijenjati detalje svog profila (ime, prezime, korisničko
    ime i e-mail adresa) te izbrisati svoj profil.

    Također, korisnik/ca može mijenjati izgled aplikacije i bez da je
    prijavljen/a u aplikaciju. Moguće je odabrati jezik aplikacije (hrvatski ili
    engleski), temu (svijetla ili tamna) i proizvoljnu boju isticanja (engl.
    \textit{accent color}).

\section{Programski alati}

  \subsection{Uređivači izvornog koda}

    \subsubsection*{\textit{VSCodium}}

      \textit{VSCodium} je program uređivač izvornog koda koji je razvio
      Microsoft 2015. g. Vrlo je popularan, funkcionalan i prilagodljiv.
      Podržava otklanjanje grešaka, ugrađenu \textit{git} kontrolu, isticanje
      sintakse i automatsko dovršavanje koda. \textit{VSCodium} je inačica
      programa \textit{VSCode} iz koje je uklonjeno telemetrijsko prikupljanje
      korisničkih podataka. Izvorni kod programa je besplatan i otvoren.

      \begin{figure}[h]
        \includegraphics[width=\textwidth]{vscodium}
        \caption{VSCodium uređivač koda s otvorenim projektom.}
      \end{figure}

      \textit{VSCodium} sam koristio kao glavni uređivač teksta za cijeli
      projekt. Omogučio mi je brzo, organizirano i funkcionalno digitalno radno
      okruženje.

      \subsubsection*{\textit{vim}}

      \textit{vim} je program uređivač izvornog koda nastao 1991. g. Za razliku
      od \textit{VSCodiuma} koji ima grafičko korisničko sučelje (GUI),
      \textit{vim} ima isključivo tekstualno korisničko sučelje (TUI).

      \textit{vim} sam koristio isključivo za uređivanje pojedinačnih datoteka
      izvan projekta i malih konfiguracijskih datoteka (npr. Apache
      konfiguracija).

  \subsection{Kontrola izvornog koda}

    \textit{git} je program za upravljanje izvornim kodom nastao 2005. g.
    Njegov kreator Linus Torvalds je ujedno i voditelj razvoja Linux
    operativnog sustava.

    Pomoću \textit{gita} sam spremao i bilježio napredak svog projekta.
    Također sam objavio izvorni kod svoje aplikacije na web sjedištu
    \textit{GitHubu}, koji nudi uslugu tzv. \textit{hostinga} izvornog koda,
    što je ujedno služilo za čuvanje sigurnosne kopije.

  \subsection{Terminal}

    \textit{kitty} je napredni Linux emulator terminala (konzola). Koristio sam
    ga za uklanjanje grešaka, praćenje izlaznih informacija \textit{Apache}
    servera, upravljanje bazom podataka slanjem naredba, za \textit{git} naredbe
    i općenito za upravljanje datotekama u projektu.

    \begin{figure}[h]
      \includegraphics[width=\textwidth]{kitty}
      \caption{kitty terminal s više otvorenih kartica i panela.}
    \end{figure}

  \subsection{\textit{Web stack}}

    Aplikacija je razvijana na klasičnom \textit{LAMP} web stogu programske
    podrške (engl. \textit{web stack}) - Linux, Apache, MariaDB i PHP. Apache je
    popularan HTTP web server. MariaDB je popularan DBMS (engl. \textit{Database
    Management System}).

  \subsection{\textit{Web} preglednici}

    \begin{wrapfigure}{r}{0.5\textwidth}
      \includegraphics[width=\linewidth]{devtools}
      \caption{Alati za web programere, preglednik Mozilla Firefox.}
    \end{wrapfigure}

    Aplikacija je testirana na dvoje najpopularnijih web preglednika: Google
    Chrome i Mozilla Firefox. Potrebno je testirati na više preglednika radi
    konzistencije izgleda sučelja jer svaki preglednik ima malo drugačiju
    implementaciju HTML, CSS i JS standarda.

    Važno je spomenuti i integrirane alate za web programere koje nude oba web
    preglednika: inspektor (engl. \textit{inspector}), konzola (engl.
    \textit{console}), spremište (engl. \textit{storage}) itd.

  \subsection{Ostalo}

    \begin{itemize}
      \item JSON linter/validator/minifier: \\ \url{https://jsonlint.com/}
      \item regular expressions 101: \\ \url{https://regex101.com/}
      \item phpMyAdmin: \\ \url{https://www.phpmyadmin.net/}
      \item Fontovi: \\ \url{https://fonts.google.com/}
      \item Ikonice: \\ \url{https://icons.getbootstrap.com/}
    \end{itemize}

\section{Programski i ostali jezici}

  \subsection{\textit{Front-end}}

  Pod \textit{front-end} dio aplikacije spadaju sve stranice web aplikacije koje
  su vidljivi krajnjem korisniku/ci. Preko njih korisnik/ca upravlja i
  prosljeđuje podatke aplikaciji.

    \subsubsection*{\textit{HTML}}

      HTML (engl. \textit{HyperText Markup Language}) je označni jezik za
      definiranje osnovne strukture i sadržaja web dokumenata. Koristim aktualnu
      inačicu i standard \textit{HTML5}.

    \subsubsection*{\textit{CSS}}

      \textit{CSS} (engl. \textit{Cascading Style Sheets}) je jezik za
      oblikovanje i stiliziranje HTML dokumenata. Omogućava definiranje položaja
      elemenata, boje, fonta, efekata i sl. Koristim aktualnu inačicu i standard
      \textit{CSS3}.

    \subsubsection*{\textit{JavaScript}}

      \textit{JavaScript} (skraćeno \textit{JS}) je programski jezik za
      interakciju s korisnikom, razmjenu podataka i kreiranje dinamične,
      funkcionalne web stranice. Jezik je visoke razine, dinamički pisan, OO
      (objektno orijentiran) i interpretiran. Valja spomenuti i DOM (engl.
      \textit{Document Object Model}) pomoću kojeg je moguće dinamički
      manipulirati HTML-om i CSS-om dokumenta.

    \begin{figure}[h]
      \centering
      \includegraphics[height=4cm]{front_end}
      \vspace{0.5cm}
      \caption{Front-end jezici.}
    \end{figure}

  \subsection{\textit{Back-end}}

    \subsubsection*{\textit{SQL (MariaDB)}}

      SQL (engl. \textit{Structured Query Language}) je deklarativni jezik koji
      služi za postavljanje upita (engl. \textit{query}) DBMS-u (bazi podataka).
      Upitima je moguće kreirati (engl. \textit{create}), čitati (engl.
      \textit{read}), ažurirati (engl. \textit{update}) i brisati (engl.
      \textit{delete}) podatke - tzv. CRUD operacije.

    \subsubsection*{\textit{PHP}}

      PHP je skriptni programski jezik koji se izvršava isključivo na
      poslužitelju. Jezik je visoke razine, dinamički pisan, OO (objektno
      orijentiran) i interpretiran. Služi za komuniciranje aplikacije s bazom
      podataka. Nakon izvršavanja na serveru, sav HTML izlaz PHP skripte šalje
      se klijentskom računalu i korisnik vidi gotovu stranicu.

    \begin{figure}[h]
      \centering
      \includegraphics[width=\textwidth]{back_end}
      \caption{Back-end programski jezici i programska podrška.}
    \end{figure}

  \subsection{Ostalo}

    \subsubsection*{\textit{JSON}}

      JSON (JavaScript Object Notation) je tekstualni format za spremanje
      podataka. JSON zapisi razumljivi su i čovjeku i računalu (programu).

    \subsubsection*{\textit{LaTeX}}

      PDF dokument elaborata napravljen je pomoću označnog jezika \textit{LaTeX}
      i prevoditelja (engl. \textit{compiler}) \textit{latexmk}. Iz izvornog
      \textit{LaTeX} koda prevoditelj generira profesionalno formatirani i
      stilizirani PDF dokument.

\section{Arhitektura baze podataka}

  \subsection{Dizajn baze podataka}

    Baza podataka sastoji se od 2 tablice: \textit{User} -- korisnici i
    \textit{Document} -- dokumenti. Veza je 1:M (1 naprema više).

    Svaki zapis entiteta jedinstven je po svojem primarnom ključu (engl.
    \textit{primary key}, oznaka \textit{PK}). PK je broj koji se inkrementira
    za svaki novi zapis tog entiteta.

    Tablice se vezuju pomoću tzv. stranih ključeva (engl. \textit{foreign key},
    oznaka \textit{FK}). \textit{FK} pokazuje na odgovarajući \textit{PK} u
    nekoj drugoj tablici, i tipovi podataka im moraju biti identični.

    Većina atributa imaju i \textit{not null constraint}, što znači da
    vrijednosti tog atributa svakog zapisa u tablici treba biti poznat
    (definiran).

    Baza podataka je projektirana u skladu s prve 3 normalne forme (1NF, 2NF,
    3NF).

    \begin{figure}[h]
      \centering
      \includegraphics[scale=0.85]{model}
      \caption{Dijagram baze podataka prikazan u \textit{phpMyAdminu}.}
    \end{figure}

    \subsubsection{Tipovi podataka}

      Važno je kvalitetno odabrati tipove podatka atributa u tablicama tako da
      se uštedi na memoriji, ali ujedno osigura i buduća prilagodba i
      proširivost funkcionalnosti baze.

      Primjer toga je način korištenja char i varchar tipova podataka. Brojevi u
      zagradi (npr. \textit{varchar(40)}) označavaju najveći broj znakova u tom
      atributu. Također su korišteni i unsigned mediumint primarni ključevi da
      se dodatno uštedi na pohrani.

    \subsubsection{Entitet \textit{User}}

      Ova tablica služi za pohranu podataka o svakom registriranom korisniku.
      Čuvaju se: korisničko ime, e-mail, ime, prezime, datum registracije i
      datum i vrijeme zadnje prijave. Također postoji i atribut
      \textit{passwordHash}, koji sprema lozinku korisničkog računa, čime se
      ostvaruje autentikacija i autorizacija korisničkih podataka.

      Valja napomenuti da se prije pohrane u bazu sve lozinke procesiraju kroz
      jednosmjerni algoritam za tzv. \textit{hashiranje}, tj. kriptira se, što
      služi kao osnovna mjera zaštite korisničkih lozinka.

    \subsubsection{Entitet \textit{Document}}

      Ova tablica služi za pohranu podataka o dokumentima, tj. ispitima i
      obrascima. Čuvaju se: naziv dokumenta, tip, vidljivost, broj dozvoljenih
      pokušaja (predaja), autor dokumenta (FK), datum i vrijeme roka predaje i
      datum izrade.

      Atribut \textit{documentJSON} je tipa \textit{JSON} i služi za pohranu
      pitanja i sadržaja od kojih se sastoji svaki dokument.

      Atribut \textit{solutionJSON} pohranjuje samo rješenja dokumenta (ako je
      taj dokument tipa ispit). Kako bi aplikacija bila povjerljiva, iz
      sigurnosnih razloga, rješenja dokumenta (odgovori) su odvojena od samih
      pitanja.

    \subsection{\textit{SQL} skripta baze podataka}

      \subsubsection{Izrada MariaDB korisnika i baze podataka}

        \lstinputlisting[language=SQL, caption={
          SQL -- Izrada MariaDB korisnika i baze podataka.}]{01_setup.sql}

      \pagebreak[4]
      \subsubsection{Izrada i povezivanje tablica}

        \lstinputlisting[language=SQL, caption={
          SQL -- Izrada i povezivanje tablica.}]{02_schema.sql}


\section{Izvedba aplikacije}

  \subsection{PHP konfiguracija}

    Ova PHP skripta se obavezno \textit{includea} na vrhu svake druge PHP
    skripte. Sadrži neke osnovne PHP postavke.

    \lstinputlisting[language=PHP, alsolanguage=HTML, caption={
      PHP -- Osnovna konfiguracija}]{config.php}

  \subsection{Recikliranje koda}

    Jedan od temeljnih principa programiranja je izbjeći lošu redundanciju u
    kodu (engl. \textit{DRY - Don't Repeat Yourself.}). Iz tog razloga sam kod
    aplikacije pisao vrlo organizirano i modularno, te pokušavao definirati
    metode i funkcije za kod koji se ponavlja.

    Npr. postoje klase u projektu \textit{UserModel} i \textit{DocumentModel}.
    \textit{UserModel} sadrži sve metode i atribute koje opisuju korisnika.
    Slično vrijedi i za \textit{DocumentModel} klasu.

    \subsubsection{\textit{Util} klasa}

      \textit{Util} klasa (engl. \textit{Utility}) sadrži statičke metode koje
      se često koriste na raznim mjestima u projektu, ali ne mogu se
      kategorizirati u klasu za sebe.

      \lstinputlisting[language=PHP, alsolanguage=HTML, caption={
          PHP -- Util klasa}]{Util.php}

    \subsubsection{\textit{DB} klasa}

      \textit{DB} klasa (engl. \textit{DataBase}) sadrži često korištene metode
      i svojstva vezane za upravljanje bazom podataka, koje su smještene u tzv.
      \textit{singleton} klasu.

      \lstinputlisting[language=PHP, caption={
          PHP -- DB klasa}]{DB.php}

      \subsubsection{Jezik aplikacije}

        Višejezičnost aplikacije je postignuta tako da su svi znakovni nizovi
        korišteni u aplikaciji organizirani u dvije slične PHP datoteke -
        lang\_en.php i lang\_hr.php. Obje datoteke sadrže asocijativno polje
        \textit{LANG} -- ključevi su isti, a vrijednosti prevedene.

        \lstinputlisting[language=PHP, caption={
          PHP -- Engleski znakovni nizovi}]{lang_en.php}

        \lstinputlisting[language=PHP, caption={
          PHP -- Hrvatski znakovni nizovi}]{lang_hr.php}

    \subsubsection{Osnovna struktura stranice}

    \begin{figure}[h]
      \includegraphics[width=\textwidth]{page_template}
      \caption{Početna stranica s popisom dokumenata.}
    \end{figure}

    Na svakoj stranici se ponavljaju zaglavlje i podnožje. Zaglavlje i podnožje
    stranice su stavljeni u zasebne datoteke (header.phtml i footer.phtml) te se
    na svakoj stranici \textit{includeaju}.

    Zaglavlje sadržava logotip aplikacije, naslov i navigator s poveznicama koje
    vode do glavnih stranica (engl. \textit{views}) aplikacije. Navigator
    mijenja svoj sadržaj ovisno o tome je li korisnik/ca prijavljen/a. Ako nije,
    prikazuju se poveznice za registraciju i prijavu. Ako je, onda su te
    poveznice sakrivene.

    \begin{figure}[h]
      \centering
      \includegraphics[scale=0.4]{header}
      \caption{Navigator kad korisnik/ca nije prijavljen/a.}
    \end{figure}

    \subsection{Korisničke postavke (\textit{front-end})}

      \begin{figure}[h]
        \begin{center}
          \begin{subfigure}{0.65\textwidth}
            \includegraphics[width=\linewidth]{page_preferences_1}
          \end{subfigure}
          \\
          \begin{subfigure}{0.65\textwidth}
            \includegraphics[width=\linewidth]{page_preferences_2}
          \end{subfigure}

          \caption{Prilagodljivost sučelja (profile.phtml)}
        \end{center}
      \end{figure}

      Izgled sučelja aplikacije je prilagodljiv korisniku.
      Korisničke postavke (engl. \textit{Preferences}) sastoje se od:

      \begin{itemize}
        \item teme (svijetla ili tamna),
        \item jezika (engleski ili hrvatsi)
        \item i boje isticanja (proizvoljna).
      \end{itemize}


    \subsection{Korisničke postavke (\textit{back-end})}

      Da bi mijenjao/la ove postavke prikaza, korisnik/ca ne mora biti
      prijavljen/a u aplikaciju, jer se ove postavke spremaju u tzv. kolačić
      (engl. \textit{cookie}). Ako kolačić ne postoji (prva posjeta \textit{web}
      sjedištu), dodjeljuju se zadane (\textit{defaultne}) postavke (svijetla
      tema, zelena boja isticanja, engleski jezik).

      \subsubsection{\textit{Preferences} klasa}

        \lstinputlisting[language=PHP, caption={
          PHP -- Preferences klasa}]{Preferences.php}

  \subsection{Registracija korisnika (\textit{front-end})}

    \subsubsection{Prikaz obrasca za registraciju}

      \begin{figure}[h]
          \begin{subfigure}{0.5\textwidth}
            \includegraphics[width=0.9\linewidth]{form_register_1}
          \end{subfigure}
          \begin{subfigure}{0.5\textwidth}
            \includegraphics[width=0.9\linewidth]{form_register_2}
          \end{subfigure}

          \caption{Obrazac za registraciju -- validacija u realnom vremenu}
      \end{figure}

      Kako bi koristio/la sve mogućnosti aplikacije, korisnik/ca mora biti
      registriran/a. To obavlja preko ovog obrasca, koji je intuitivan jer ima
      ugrađenu tzv. \textit{front-end} validaciju.

      Također, desno od polja za unos lozinke nalazi se gumb za prikazivanje
      lozinke. Lozinka se prikazuje samo kad korisnik/ca drži gumb pritisnutim,
      inače je lozinka skrivena. To je postignuto trivijalnim JS kodom.

      Sva polja moraju biti popunjena i ispravna. Ukoliko neko polje nije ispravno,
      na dnu obrasca se u realnom vremenu (prije podnašanja obrasca) prikazuje
      povratna informacija korisniku/ci te mu/joj nije dozvoljeno podnašanje
      obrasca.

      Ukoliko je ispravno, polje se blago osjenča zelenom bojom. Ovo su uvjeti za
      ispravnost polja:

      \begin{itemize}
        \item Korisničko ime: Koristite velika i mala slova bez dijakritičkih
        znakova, brojeve i donje crte, duljine najmanje 4 znakova.
        \item Lozinka: Duljine barem 8 znakova, barem 1 veliko slovo i 1 broj.
        \item Ime i prezime: Duljine najmanje 3 znakova, bez brojeva.
        \item Korisničko ime i e-mail adresa ne smiju biti zauzeti
      \end{itemize}

      Valja napomenuti da su tijekom dizajniranja \textit{front-enda} uzete u
      obzir osnovne prakse pristupačnosti (engl. \textit{accessibility}):
      autofocus atributi, korištenje label elemenata koji su povezani sa svojim
      poljima za unos, te je za svako polje sačuvan tabindex atribut.

      \subsubsection{HTML/PHP struktura obrasca registracije}

        \lstinputlisting[language=PHP, alsolanguage=HTML, caption={
          HTML/PHP -- Obrazac za registraciju.}]{form_sign_up.phtml}

      \subsubsection{JS kod za validaciju}

        \lstinputlisting[language=JavaScript, caption={
          JS -- Live validacija forme.}]{form_validation.js}

  \subsection{Registracija korisnika (\textit{back-end})}

    \textit{Front-end} validaciju vrlo je lako zaobići, stoga je iz
    sigurnosnih razloga potrebno obaviti validaciju i u \textit{back-end}
    dijelu aplikacije.

    \subsubsection{Obrada obrasca za registraciju}

      \lstinputlisting[language=PHP, caption={
        PHP -- Obrada obrasca za registraciju.}]{sign_up.php}

    \subsubsection{Spremanje korisnika u bazu podataka}

      \lstinputlisting[language=PHP, alsolanguage=HTML, caption={
        PHP/SQL -- Spremanje korisnika u bazu podataka.}]{UserModelsignUp.php}

  \subsection{Prijava korisnika}

    \begin{figure}[h]
      \begin{subfigure}{0.5\textwidth}
        \includegraphics[width=0.9\linewidth]{form_login_1}
      \end{subfigure}
      \begin{subfigure}{0.5\textwidth}
        \includegraphics[width=0.9\linewidth]{form_login_2}
      \end{subfigure}

      \caption{Obrazac za prijavu}
    \end{figure}

    Korisnik/ca se može prijaviti ili korisničkim imenom ili e-mail adresom.
    HTML kod i PHP kod za validaciju je nepotrebno pokazivati jer su vrlo slični
    kodu svakog drugog obrasca u projektu.

    Ako je lozinka pogrešna, prikazuje se greška korisniku na dnu obrasca.

    \lstinputlisting[language=PHP, alsolanguage=HTML, caption={
        PHP -- Prijava korisnika}]{UserModelLogIn.php}

  \subsection{Profil korisnika}

    \begin{figure}[h]
      \includegraphics[width=\textwidth]{page_profile}
      \caption{Stranica s profilom korisnika.}
    \end{figure}

    Na ovoj stranici nalaze se detalji korisničkog računa. Također, ako je
    korisnik/ca prijavljen/a, prikazuju se gumbi za odjavu, uređivanje detalja
    profila i brisanje samog profila.

    Klikom na gumb za odjavu uništava se sesija i preusmjerava se na obrazac za
    prijavu.

    \lstinputlisting[language=PHP, alsolanguage=HTML, caption={
        PHP -- Odjava korisnika.}]{log_out.php}

    Klikom na gumb za brisanje računa otvara se skočni modalni prozor (engl.
    \textit{overlay}) za potvrdu brisanja. Ako korisnik potvrdi, briše se račun
    i svi dokumenti korisnika.

    \lstinputlisting[language=PHP, alsolanguage=HTML, caption={
      PHP/SQL -- Brisanje korisnika.}]{UserModelDelete.php}

    \begin{figure}[h]
      \centering
      \includegraphics[width=0.5\textwidth]{delete_profile}
      \caption{Skočni modalni prozor za potvrdu brisanja računa.}
    \end{figure}

    Klikom na gumb za uređivanje profila, otvara se isti obrazac kao za
    registraciju korisnika (osim polja za lozinku) -- HTML struktura i kod za
    validaciju se reciklira. Razlika je u tome što se ovaj obrazac prikazuje u
    \textit{overlayu}. Polja se popune s vrijednostima u bazi podataka.

    \begin{figure}[h]
      \centering
      \includegraphics[width=0.5\textwidth]{form_edit_profile}
      \caption{Skočni modalni prozor za uređivanje profila.}
    \end{figure}

  \subsection{Početna stranica (\textit{front-end})}

    \subsubsection{Popis dokumenata}

      Na početnoj stranici su tabelarno prikazani svi dokumenti korisnika te
      njihovi detalji. Klikom na poveznicu u prvom stupcu tablice preusmjerava na
      stranicu detalja dokumenata, gdje su prikazani svi detalji dokumenta.
      Prelaskom miša (engl. \textit{hover}) preko datuma roka (ako je definiran)
      prikazuje se u malom prozoru (engl. \textit{tooltip}) \textit{relativan}
      datum.

      \begin{figure}[h]
        \centering
        \begin{subfigure}{\textwidth}
          \includegraphics[width=0.9\linewidth]{documents_list}
        \end{subfigure}
        \\
        \begin{subfigure}{\textwidth}
          \includegraphics[width=0.9\linewidth]{documents_list_en}
        \end{subfigure}

        \caption{Popis dokumenata; višejezičnost.}
      \end{figure}


\section{Zaključak}

  Elaborat završnog rada dokumentira arhitekturu i programsku izvedbu web
  aplikacije za kreiranje i rješavanje ispita i obrazaca. Korisnik/ca može nakon
  prijave kreirati svoje dokumente i rješavati tuđe. Korisnik/ca vidi popise
  vlastitih i tuđih dokumenata i rješenja tih dokumenata. Moguće je uređivanje
  detalja i brisanje vlastitog profile i vlastitih dokumenata. Korisnik/ca može
  prilagoditi sučelje aplikacije svojim potrebama. Sučelje aplikacije je
  intuitivno i pristupačno.

  Prilikom izvedbe web aplikacije korišteni su moderni JavaScript API-ji: Fetch,
  DOM i Intl. PHP nudi MySQLi API za upravljanje bazom podataka. U PHP-u sam
  također koristio kolačiće i varijable sesije. I JavaScript i PHP kod pretežito
  koriste OOP model programiranja.

  Zahvaljujući modularnosti koda i kvalitetnim komentarima, aplikaciju je
  jednostavno proširiti. U budućnosti bih preveo aplikaciju na više jezika,
  dodao prijavu pomoću Google API-ja, dodao još različitih tipova pitanja te
  dotjerao \textit{front-end} da bude još profesionalnijeg izgleda.

  Nastojao sam pisati dobro organiziran, siguran, moderan, kvalitetno
  dokumentiran, efikasan i modularan izvorni kod -- koliko je god bilo moguće.
  Ovo mi je najopširniji programerski projekt do sad. Naučio sam jako puno
  korisnih stvari vezanih uz struku i predmet te uveliko proširio znanje i
  vještine stečene u školi.

\listoffigures

\lstlistoflistings
\addcontentsline{toc}{section}{\lstlistlistingname}

\section*{Literatura}
\addcontentsline{toc}{section}{Literatura}

  \begin{itemize}
    \item \textit{Stack Overflow} - web sjedište sa pitanjima i odgovorima za
      profesionalne programere: \url{https://stackoverflow.com/}
    \item \textit{PHP} priručnik: \url{https://www.php.net/manual/en/}
    \item \textit{MDN Web Docs} - priručnik za \textit{HTML}, \textit{CSS} i
      \textit{JavaScript}: \\\url{https://developer.mozilla.org/en-US/}
    \item \textit{MariaDB SQL} dokumentacija:
      \url{https://mariadb.com/kb/en/documentation/}
    \item \textit{Apache} dokumentacija: \url{https://httpd.apache.org/docs/2.4/}
  \end{itemize}

\cleardoublepage
\section*{Konzultacijski list učenika \hfill Razredni odjel: 4.AT}


\begin{table}[h]
      \subsubsection*{Konzultacije s mentorom:}

      \def\arraystretch{1.6}
      \footnotesize

      \centering
      \begin{tabularx}{\textwidth}{|r|X|l|l|}
      \cline{1-4}
      \textbf{RB} & \textbf{Sadržaj (bilješke o napredovanju)} & \textbf{Potpis mentora} & \textbf{Datum} \\ \cline{1-4}
      1. & Mentor prihvatio predloženu temu rada  &                & 27.10.2023. \\ \cline{1-4}
      2. & Poslana poveznica za GitHub repozitorij  &                & 15.01.2024. \\ \cline{1-4}
      3. & Konfiguracija hostinga &                & 23.02.2024. \\ \cline{1-4}
      4. & Savjeti za pisanje elaborata  &   &  25.04.2024. \\ \cline{1-4}
      5. & Slanje prve inačice elaborata mentoru na uvid  &                &   01.05.2024.    \\ \cline{1-4}
      \end{tabularx}
    \end{table}

    \begin{center}
      \vspace{-0.5cm}
      \begin{tabular}{@{}l@{}}
        Datum predaje rada: \rule{3cm}{0.5pt} \\
        Potpis mentora: \rule{4cm}{0.5pt} \scriptsize{(mentor je prihvatio izradbu)}\normalsize \\
        Ocjena pisanog rada: \rule{4cm}{0.5pt} \\
        Datum obrane rada: \rule{3cm}{0.5pt} \\
        Ocjena obrane rada: \rule{4cm}{0.5pt} \\
        \textbf{Konačna ocjena: \rule{4cm}{0.5pt}}
        \vspace{-0.5cm}
      \end{tabular}
      % \mbox{Datum predaje rada: \rule{4cm}{0.5pt} \hfill Potpis mentora: \rule{4cm}{0.5pt} }

      % \mbox{Ocjena pisanog rada: \rule{6cm}{0.5pt}}


      % \mbox{Konačna ocjena: \rule{6cm}{0.5pt}}
    \end{center}

  \subsubsection*{Povjerenstvo:}

    % Set the width of the text into \mytextwidth
    \settowidth{\mytextwidth}{mentor: }

    % Define the length of the line
    \setlength{\myline}{\mytextwidth}

    \begin{enumerate}
      \item mentor: \rule{6.5cm}{0.5pt}
      \item \rule{\myline+6.5cm}{0.5pt}
      \item \rule{\myline+6.5cm}{0.5pt}
      \item \rule{\myline+6.5cm}{0.5pt}
      \item \rule{\myline+6.5cm}{0.5pt}
    \end{enumerate}

  \subsubsection*{Komentar:}

    \framebox[\textwidth][c]{\rule{0pt}{4.1cm}}


\end{document}
